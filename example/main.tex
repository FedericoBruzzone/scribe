\documentclass[
    lineno, % Enable line numbers.
    letterpaper, % Use 'a4paper' for A4 size.
    twocolumn, % Uncomment for two-column layout.
]{scribe} 

\usepackage{lipsum} % For generating dummy text

% Set right header mark (e.g., your name)
\scribesetrightmark{Federico Bruzzone} 

% Define different aliases (along with colors) for comments
\scribedefinecomment{fb}{orange}
\scribedefinecomment{jd}{blue}
% ...

% Uncomment the following line to hide comments in the final document.
% \scribeshowcommentsfalse 

\title{\textbf{An example paper using the scribe document class}}

\author{
    Federico Bruzzone \orcidlink{0000-0002-8701-8853} \\
    Computer Science Department \\
    Università degli Studi di Milano \\
    \href{mailto:federico.bruzzone@unimi.it}{\texttt{federico.bruzzone@unimi.it}} \\
    \url{https://federicobruzzone.github.io/} 
}

\date{September 2025}

\begin{document}


\twocolumn[
	\maketitle

	\begin{abstract}
		Lorem ipsum dolor sit amet, consectetuer adipiscing elit. Ut purus elit, vestibulum ut, placerat ac, adi- piscing vitae, felis. Curabitur dictum gravida mau- ris. Nam arcu libero, nonummy eget, consectetuer id, vulputate a, magna. Donec vehicula augue eu ne- que. Pellentesque habitant morbi tristique senectus et netus et malesuada fames ac turpis egestas. Mau- ris ut leo. Cras viverra metus rhoncus sem. Nulla et lectus vestibulum urna fringilla ultrices. Phasel- lus eu tellus sit amet tortor gravida placerat. Integer sapien est, iaculis in, pretium quis, viverra ac, nunc. Praesent eget sem vel leo ultrices bibendum. Aenean faucibus. Morbi dolor nulla, malesuada eu, pulvinar at, mollis ac, nulla. Curabitur auctor semper nulla. Donec varius orci eget risus. Duis nibh mi, congue eu, accumsan eleifend, sagittis quis, diam. Duis eget orci sit amet orci dignissim rutrum.
	\end{abstract}

	\begin{indexterms}
		foo, bar, baz, qux.
	\end{indexterms}
]


% ==============================
% ========== Sections ==========
% ==============================
\section{Introduction}\label{sec:introduction}

\lipsum[1]~\cite{Scribe25}.\footnote{This is a footnote.}
\begin{equation*}
	fact(n) = \begin{cases}
		1                 & \text{if } n = 0 \\
		n \cdot fact(n-1) & \text{if } n > 0
	\end{cases}
\end{equation*}
\fbcomment{This is a comment in orange from Federico Bruzzone.}
\lipsum[1]
\jdcomment{This is another comment in blue from Jhon Doe.}
\lipsum[2]



\bibliography{local}

\end{document}

